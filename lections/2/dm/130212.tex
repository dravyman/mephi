\documentclass[12pt]{article}
%\usepackage{ucs}
\usepackage[utf8x]{inputenc} % Включаем поддержку UTF8
\usepackage[russian]{babel}  % Включаем пакет для поддержки русского языка
\usepackage{amsmath}
\usepackage{amssymb}

\title{Дискретная математика}

\date{}
\author{abcdw}

\begin{document}
    \maketitle
    Лектор: Гусев. \newline

    Учебник: Спросить. \newline
    Задачник: Виленкин. \newline

    Основные понятия. \newline
    Комбинаторика - раздел математики, изучающий приемы нахождения числа различных комбинаций, составленных при определенных условиях. \newline
    Сочетания - соеденения в определенном порядке. \newline
    Сложный замысел - система прием для достижения чего-либо. \newline
    Уловка - договоренность. \newline
    Хаос - пространство неупорядоченных возможностей. \newline
    Комбинаторный - зависящий от распределения или от размещения. \newline
    Принятие решений - осуществление искомого выбора. \newline

    Проблемы. \newline
    1. Проблема существования.
    2. Перечеслительная проблема.
    3. Выбор наилучшего.

    Задачи: \newline
    Укладка: $\bigcup\limits_i T_i \subseteq S$ \newline
    Покрытие: $\bigcup\limits_i T_i \supseteq S, T_i \cap T_j \not = \varnothing$ \newline
    Разбиение: $\bigcup\limits_i T_i = S, T_i \cap T_j = \varnothing$ \newline

    Выборка - и процесс и результат. \newline
    n-множество. \newline
    r-выборка. \newline
    1. Упорядоченная. \newline
        a. С повторениями $\overline P_n^r$ \newline
        b. Без. $P_n^r, r \leq n$ \newline
    2. Неупорядоченная. \newline
        a. С повторениями $\overline C_n^r$ \newline
        b. Без. $C_n^r, r \leq n$ \newline

    n-выборка из k-множества. \newline
    Заполнение из одного множества других. С учетом вида элементов и их числа, вида урн и их вместимости, порядка элемента, порядка ящика. \newline
    <N, f, K> - морфизм. \newline
    Map(N, K) - произвольное отображение. = \{$f: N \rightarrow K,$ f - произвольная\} \newline
    Sur(N, K) - сюрьективное отображение. \newline
    Inj(N, K) - инъективное отображение. \newline
    Bij(N, K) - биективное отображение. \newline

    N, K - конечные. \newline
    1. Sur(N, K) $|N| \geq |K|$ \newline
    2. Inj(N, K) $|N| \leq |K|$ \newline
    3. Bij(N, K) $|N| = |K|$ \newline

    Для отображения конечного множества на себя Sur, Inj, Bij совпадают. \newline

    Примеры:

    N = K = R
    1. $f: x \rightarrow 3x - 2$ - Sur. \newline
    2. $f: x \rightarrow x^2$ - не Sur. \newline

    $N = R, K = R_+$ \newline
    1. $f: x \rightarrow e^x$ - Inj. \newline

    Заполнение ящиков(урн). Все элементы K упорядочиваем и рассматриваем их как ящики. \newline
    Интерпретация комбинаторных операций - составление слов из букв множества K, помеченных индексом из N. \newline
    Составление слов длины n = |N| из букв алфавита K, снабженных индексами из N. \newline
    $N = \{1 < 2 < 3\}$ \newline
    $K = \{a < b < c\}$ \newline

\end{document}
